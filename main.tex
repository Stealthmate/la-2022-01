\newif\ifkokyuroku\kokyurokufalse

%%%%%%%%%%%%%%%
% 講究録原稿の場合は次の一行を有効にする
%\kokyurokutrue

\ifkokyuroku
\documentclass[11pt,a4paper]{jarticle}
\else
\documentclass[10pt,a4paper,twocolumn]{jarticle}
\fi

%%%%%%%%%
% タイトル
\title{前方に人がいることを許した自転車共有問題}
%%%%%%%%%

\ifkokyuroku
\usepackage{authblk}
\renewcommand\Authsep{\qquad}
\renewcommand\Authand{\qquad}
\renewcommand\Authands{\qquad}

%%%%%%%%%
% 講究録用(著者情報)
\author[1]{ハララノフ ヴァレリ   Haralanov Valeri }
\author[2]{著者 い    author asdf B}
\author[1]{著者 う  author asdf C}
\author[1]{\\著者 え    author D}
\author[2]{著者 お    author asdf E}

\affil[1]{
 九州大学大学院システム情報科学府 \authorcr
  Graduate School of Information Science and Electrical Engineering, Kyushu University
}
\affil[2]{
 九州大学工学部 \authorcr
 School of Engineering, Kyushu University
}
% end of 講究録用
%%%%%%%%%

\setlength{\topmargin}{-0.3cm}
\setlength{\textheight}{23cm}
\setlength{\oddsidemargin}{0.5cm}
\setlength{\textwidth}{15.0cm}

\else

%%%%%%%%%
% 予稿用(講演番号と著者情報)
\def\kouenbangou{88} %数字を自分の講演番号に書き換えて下さい.
\author{
	Haralanov Valeri \thanks{埼玉大学工学部情報工学科} \and
	Yamada Toshinori \thanks{埼玉大学大学院理工学研究科数理電子情報部門} \and
	% 著者3\footnotemark[1]
	%「\footnotemark[N]」で第N著者と同じマークが付きます.
}
% end of 予稿用
%%%%%%%%%

\makeatletter
\def\ps@LAheadings{
	\def\@oddhead{}
	\def\@evenhead{}
	\def\@evenfoot{\hfil \kouenbangou\,--\,\thepage\hfill}
	\def\@oddfoot{\hfil \kouenbangou\,--\,\thepage\hfil}
}
\def\ps@LAtitleheadings{
	\def\@oddhead{2021年度冬のLAシンポジウム\,[\kouenbangou]\hfil}
	\def\@evenhead{}
	\def\@evenfoot{\hfil \kouenbangou\,--\,\thepage\hfill}
	\def\@oddfoot{\hfil \kouenbangou\,--\,\thepage\hfil}
}
\makeatother

\pagestyle{LAheadings}
\fi

\date{}

%%%%%%%%%
% ダミーの文章、および定規を生成する記述です。原稿作成の際は消去して構いません
\usepackage{blindtext}
\usepackage[type=none]{fgruler}
%%%%%%%%%

%%%% Custom
\usepackage{amsmath}
\usepackage{amsfonts}
\usepackage[backend=biber]{biblatex}
\usepackage{amsthm}
\usepackage[boxruled,linesnumbered,noend]{algorithm2e}
\usepackage[toc,page]{appendix}
\usepackage{tabularx,lipsum,environ}
\renewcommand\appendixpagename{付録}

\DeclareMathOperator*{\argmin}{\arg\min}

% Environments
\newtheorem{definition}{定義}
\newcommand\defref[1]{定義\ref{def:#1}}
\newtheorem{theorem}{定理}
\newcommand\thmref[1]{定理\ref{theorem:#1}}
\newtheorem{lemma}{補題}
\newcommand\lemref[1]{補題\ref{lemma:#1}}
\newtheorem{corollary}{系}
\newcommand\corref[1]{系\ref{corollary:#1}}
\newtheorem{proposition}{命題}
\newcommand\propref[1]{命題\ref{proposition:#1}}
\newtheorem{condition}{条件}
\newcommand\condref[1]{条件\ref{condition:#1}}
\newcommand\algref[1]{アルゴリズム \ref{alg:#1}}
\newcommand\lineref[1]{第 \ref{alg:line:#1} 行}

\def\N{\mathbb{N}}
\newcommand\emphasize[1]{\textit{#1}}
\def\st{\,s.t.\,}
\def\pause{,\,}


%% https://tex.stackexchange.com/questions/255673/problem-definition-environment
\makeatletter
\newcommand{\problemtitle}[1]{\gdef\@problemtitle{#1}}% Store problem title
\newcommand{\probleminput}[1]{\gdef\@probleminput{#1}}% Store problem input
\newcommand{\problemoutput}[1]{\gdef\@problemoutput{#1}}% Store problem question
\newcommand{\problemobjective}[1]{\gdef\@problemobjective{#1}}
\NewEnviron{problem}{
  \problemtitle{}\probleminput{}\problemoutput{}\problemobjective{}% Default input is empty
  \BODY% Parse input
  \par\addvspace{.5\baselineskip}
  \noindent
  \begin{tabularx}{\linewidth}{@{\hspace{\parindent}} l X c}
    \multicolumn{2}{@{\hspace{\parindent}}l}{\@problemtitle} \\% Title
    \textbf{入力:} & \@probleminput \\% Input
    \textbf{出力:} & \@problemoutput \\% Question
    \textbf{目的:} & \@problemobjective% Question
  \end{tabularx}
  \par\addvspace{.5\baselineskip}
}
\makeatother

%%%%

\addbibresource{refs.bib}


\let\oldnl\nl% Store \nl in \oldnl
\newcommand\nonl{%
  \renewcommand{\nl}{\let\nl\oldnl}}% Remove line number for one line

\begin{document}
\maketitle

\ifkokyuroku
\else
\thispagestyle{LAtitleheadings}
\fi

\section{はじめに}
TODO
\section{自転車共有問題 (BS)}
自転車共有問題を次のように定義する\parencite{czyzowicz}.まず,入力として与えられる情報は
\begin{itemize}
\item $m \in \N$: 人の数,
\item $U \in (0, 1)^{b}$: それぞれの自転車の速度が $v_i$ のとき, $u_i = \frac{1}{v_i}$ をその逆数として列にまとめたもの.
\end{itemize}
なお, $b$ は自転車の数を表すことに注意する.さらに, $b < m$ となるような入力のみを考えることとし,  $U$ が昇順にソートされているとする.

はじめに全ての人と自転車が点 0 (以降\emphasize{出発点}) に配置されているとする.人は速度 1 で\emphasize{歩く}か,とある自転車 $i$ に乗って速度 $v_i$ で移動することができる.自転車には同時に一人しか乗ることができなく,その人が自転車に乗るためには人も自転車も同時に同じ場所にいなければならない.さらに,人は任意の時点で自転車を降りることができる.また,人が乗っていない自転車は移動することができない.

BS の目標は全ての人及び自転車が点 1 (以降\emphasize{到着点}) まで最も早く移動できるような\emphasize{スケジュール}を組み立てることである.なお自転車も到着地点まで移動しなければならないことに注意する.自転車が人より少ないので,最適なスケジュールを求めるのは自明ではなく,人がどうにか自転車をうまく共有するように設計しなければならない. \textcite{czyzowicz} により提案されたアルゴリズムはまさにそのような特別な共有パターンを活用している.

自転車の共有という概念は行列として表すことができる.区間 $[0, 1]$ を $n$ 個の小区間 $x_j$ に分け,人 $i$ が小区間 $j$ で乗った自転車の番号を $M_{i,j}$ と置き, $M$ をスケジュール行列と呼ぶ.ただし,人 $i$ が徒歩で移動した小区間では $M_{i,j} = 0$ とする.しかし $M$ は実際の計算では少し使いづらいので,自転車の番号ではなく速度の逆数を格納した行列 $\widetilde{M}_{i, j} = u_{M_{i, j}}$ も定義しておく.それぞれの小区間の長さをベクトル $X \in [0, 1]^{n}$ にまとめ,順序対 $(X, M)$ をスケジュールと呼ぶ.さらに,スケジュールに対し以下の量を定義する.
\begin{itemize}
  \item 人 $i$ が小区間 $j$ の終点に到着するのに必要な時間
  \[
    t_{i,j}(X, M) = \sum_{k = 1}^{k \leq j} X_j \widetilde M_{i, j}.
  \]
\item 人 $i$ が到着点に到着するのに必要な時間
  \[
    t_i(X, M) = t_{i,n}(X, M) = \sum_{k = 1}^{k \leq n} X_j \widetilde M_{i, j}.
  \]
\item 全員が到着点に到着するのに必要な時間
  \[
    \tau(X, M) = \max_{i} t_i(X, M).
  \]
\end{itemize}
また,特定のスケジュールに関係なく,とある BS の入力に対し最適な時間を $\bar\tau(m, U)$ として表す.

これらの定義を用いていくつかの条件を加えることで $M$ に対する線形計画法で $X$ と $\tau$ を求めることができる.したがって BS の鍵となるのは $M$ の計算である. \textcite{czyzowicz} は $\bar\tau$ の下界を 2 つ示し,いずれかが必ず満たされるようなスケジュール行列の計算方法を示した.それぞれの下界は以下の補題として定義する.

\begin{lemma}
  \begin{equation}
    \bar\tau(m, U) \geq u_b.
  \end{equation}
\end{lemma}
\begin{proof}
  自転車も到着点まで移動しなければならないが,それぞれの移動速度が決まってあるので一番遅い自転車が全区間を移動する時間は必ずかかる.
\end{proof}

\begin{lemma}\label{lemma:lower-bound-bs}
  \begin{equation}
    \bar\tau(m, U) \geq T(m, U) = 1 - \frac{1}{m}\sum_{j = 1}^b(1 - u_j)
  \end{equation}
\end{lemma}
\begin{proof}
  各人が止まることなく常に歩いているもしくは自転車に乗って動いていると仮定すれば, $T(m, U)$ は全員の移動時間の平均値を表す.他方最適なスケジュール $(M, X)$ に対し $\bar\tau(m, U) = \tau(M, X) = \max_i t_i$ となるのが,最大値が平均値以上でなければならないことから主張が成り立つ.
\end{proof}
\lemref{lower-bound-bs} では人が常に動いているというのと,後退をしないという仮定が必要であるが,\textcite{czyzowicz} はそのどちらを許したとしてもより早い到着時間が得られないことを示している.以下の主張は簡単であるが,解法アルゴリズムに対し重要なので敢えて述べておく.

\begin{corollary}\label{corollary:lower-bound-bs-equality}
  全員が同時に到着するときかつそのときに限り, $\bar\tau(m, U) = T(m, U)$.
\end{corollary}

BS を解くアルゴリズムは \lemref{lower-bound-bs} 及び \corref{lower-bound-bs-equality} を活用したものである.その概ねの挙動を以下に示す.

$u_b \leq T(m, U)$ の場合,一部の人に順番に先に自転車に乗ってもらって途中で降りて歩いてもらう.なお空間的に自転車の位置が自転車の速さと同順であり,速い自転車が先にある状態を維持し,自転車を降りた人達が歩行するときに同時に同じところを歩く状態を作る.全員が一回自転車に乗ったあと,最後に乗っている人が先の歩行者に追いついた時点でそのグループに加わり,次の人が追いつくまでの区間では歩行者と追いついた自転車一台でまたグループとして動いてもらうことを考える.これはつまりより小さい入力に対して同じ問題を解くことを意味する.なおグループの動きの性質として,全員が同じ場所から同時出発をすると,次に人が追いついたときにまた全員が同じときに同じ場所にいることが保証される.この性質を用いて後ろの人と自転車をどんどん吸収していき,一番遅い自転車に乗っている一番後ろの人がちょうど到着点に他の人と合流するようなスケジュールを組むことで $\bar\tau(m, U) = T(m, U)$ を満たすスケジュールを得ることができる.

一方 $T(m, U) < u_b$ の場合,遅い自転車から取り除いていくと $u_k \leq T(m - b + k, U_k) \leq u_b$ を満たすグループが作れることが保証される.ただし $U_k$ は $k$ 番目までの自転車のみを含んだ列である.このとき,小さいグループを上記の方法で動かし,余った自転車は余った人に全区間を走ってもらうことで $u_b$ に乗っている人が最後に到着するようなスケジュールを作ることができ, $\bar\tau(m, U) = u_b$ となる.
\section{前方に人がいることを許した自転車共有問題 (FSABS) }
\subsection{問題設定と定義}
FSABS を定義する前に 2 つの補助的概念を導入する.

まず,各人の初期位置を表すための配列 $A \in {(0, 1)}^{m}$ を考え, $A$ の要素が全員分あることに注意する.理論的には前方にいる人の初期位置だけで十分であるが,以降で紹介するアルゴリズムの動作の都合上,全員分の初期位置を用意する方が扱いやすいため,ここでは出発点にいる人の初期位置を 0 とし,その人達の初期位置を含めた $A$ を考える.

BS の出力は各人が各小区間で使用した自転車の番号を表す行列 $M$ と各小区間の長さを表すベクトル $X$ と定義した. FSABS の出力にもこのような形を採用することはできるが,上記同様に解法アルゴリズムの動作の都合上,次のような形の出力を考える. $m$ 個の要素からなる配列 $S$ を考え,それぞれの要素がそれぞれの人に対応するとする.人 $i$ に対して $S_i$ の値は $n_i$ 次元配列であり,それの各要素が $(\alpha_{i,j}, \beta_{i,j})$ のような順序対であり,人 $i$ が自転車 $\beta_{i,j}$ で距離 $\alpha_{i,j}$ を連続移動したことを意味する.ただし,歩行したならば $\beta_{i,j} = 0$ と置き, $0 < j < n_i$ とする.このような $S$ を自由スケジュールと呼ぶ.また, $S_i$ のことを $S$ の行と呼ぶ.

$S_i$ の各要素に対応する $[0, 1]$ 上の始点と終点をそれぞれ $\rho_{i,j}$ 及び $\sigma_{i,j}$ で表すことができる.
\begin{align}
  \rho_{i,j} &= A_i + \sum_{k=1}^{k < j}\alpha_{i, k} \\
  \sigma_{i,j} &= \rho_{i,j} + \alpha_{i, k}
\end{align}
これらをまとめて,$\alpha_{i,j}$ に対応する小区間を $\chi_{i,j} = (\rho_{i,j}, \sigma_{i,j})$ と表す.

$S$ において人 $i$ が点 $x \geq A_i$ に到達する時間を $t^{\prime}_i(x)$ とする. $t^{\prime}_i(x)$ の値は $S_i$ の要素の線形和として求めることができるが,複雑な表記を必要とするためここでは省略する.全体の到着時間は BS と同様にそれぞれの人の到着時間の最大値となるので, $S$ における到着時間を
\begin{equation}
  \tau^\prime(S) = \max_i t^{\prime}_i(1)
\end{equation}
と定義できる.また, $S$ において人 $i$ が移動した距離を
\begin{equation}
  L(S_i) = \sum_{j=1}^{n_i} \alpha_{i,j}
\end{equation}
と表す.

次に $S$ が実行可能解であるために \textcite{czyzowicz} と同様な条件を述べる.
\begin{definition}\label{definition:fsabs-feasible-schedule}
  入力 $(m, U, A)$ と上述の構造を持つ自由スケジュール $S$ に対し,以下が成り立つとき且つそのときに限り $S$ を実行可能な自由スケジュールと呼ぶ.
  \begin{enumerate}
  \item $(\forall i\st 1 \leq i \leq m)\quad L(S_i) = 1 - A_i$.
  \item $\beta_{i,j} \neq 0$ であるならば $\sigma_{i^\prime, j^\prime} = \rho_{i,j} \text{ かつ } \beta_{i,j} = \beta_{i^\prime, j^\prime}$ を満たす $i^\prime$, $j^\prime$ が存在する.
  \item $\beta_{i,j} \neq 0$ であるならば $\chi_{i,j} \cap \chi_{i^\prime,j^\prime} \neq \emptyset$ となるような $i^\prime$ 及び $j^\prime$ に対し $\beta_{i,j} \neq \beta_{i^\prime,j ^\prime}$.
  \item $\beta_{i,j} = \beta_{i^\prime, j^\prime}$ 及び $\sigma_{i,j} \leq \rho_{i^\prime, j^{\prime}}$ ならば $t^{\prime}_i(\sigma_{i, j}) \leq t^{\prime}_{i^\prime}(\rho_{i^\prime, j^\prime})$.
  \item $\forall j \st 1 \leq j \leq b\pause \exists i \st \beta_{i,n_i} = j$.
  \end{enumerate}
\end{definition}

以上の内容をまとめて, FSABS を次のように定義する.
\begin{problem}
  \problemtitle{FSABS}
  \probleminput{
    $m \in \N$: 人の数. \newline
    $U \in (0, 1)^{b}$: 自転車の速度の逆数を格納した昇順配列. \newline
    $A \in (0, 1)^{m}$: 各人の初期値を格納した昇順配列.
  }
  \problemoutput{
    $S$: 実行可能な自由スケジュール.
  }
  \problemobjective{
    $\tau^\prime(S)$ を最小化すること.
  }
\end{problem}
また,以降の議論では具体的な出力に関係なく,とある入力に対する最適な到着時間を $\bar\tau^\prime(m, U, A)$ と表す.

解法アルゴリズムでは人をグループに分け,それぞれの移動を独立に考えてからそれぞれの自由スケジュールを組み合わせるという操作を頻用する.この操作を「縦に組み合わせる」と名付け,以下の命題でその正当性を保証する.
\begin{proposition}\label{proposition:vertical-composition}
  $(m_1, U_1 = [u_1, u_2, \ldots, u_{b_1}], A_1)$ 及び $(m_2, U_2 = [u_{b_1+1}, u_{b_1+2}, \ldots, u_{b_1+b_2}], A_2)$ をそれぞれ FSABS の入力とし, $S^{(1)}$ 及び $S^{(2)}$ をそれぞれに対する実行可能解とする. $A_1$ 及び $A_2$ の要素を $A_3$ にまとめ,昇順に並べることで新しくできた人の番号を用いて $S^{(1)}$ 及び $S^{(2)}$ の行を適切な順番に並べた $S^{(3)}$ を構成する.このとき, $S^{(3)}$ は入力 $(m_1 + m_2, U_1 + U_2, A^{(3)}$ に対する実行可能解となり,その到着時間が
  \begin{equation}
    \tau^\prime(S^{(3)}) = \max \{ \tau^\prime(S^{(1)}), \tau^\prime(S^{(2)}) \}
  \end{equation}
  となる.
\end{proposition}
\begin{proof}
  $S^{(1)}$ 及び $S^{(2)}$ は独立に実行可能解であり,且つお互いの自転車を共有することがないので \defref{fsabs-feasible-schedule} の条件は自明に満たされる.
\end{proof}

最後に, FSABS が線形スケール化可能であることを主張しておく.
\begin{claim}\label{claim:fsabs-scalable}
  FSABS の入力 $(m, U, A)$ に対し $S$ が実行可能解であるとする. \defref{fsabs-feasible-schedule} や FSABS の入力に対する条件を適切に修正し,区間 $[0, 1]$ ではなく $[0, a]$ に対し問題を定義したとする.以下で与えられる $A^\prime$ 及び $S^\prime$ に対し, $S^\prime$ は入力 $(m, U, A^\prime$ に対する実行可能解となり, $\tau^\prime(S) = a\tau^\prime(S^\prime)$ となる.
  \begin{align}
    A^\prime_i &= aA_i \\
    (\alpha^\prime_{i,j},\beta^\prime_{i,j}) &= (a\alpha_{i,j},\beta_{i,j})
  \end{align}
  ただし, $(\alpha_{i,j},\beta_{i,j})$ を $S$ の元とし, $(\alpha^\prime_{i,j},\beta^\prime_{i,j})$ を $S^\prime$ の元とする.
\end{claim}
\subsection{preliminary observations {\color{red}修正が必要}}
本節では FSABS と BS の解の形の互換性について説明した上で, FSABS の下界を定める.

問題の定義上, BS のインスタンス集合は意味的に FSABS のインスタンス集合の部分集合である.すなわち,任意の BS のインスタンス $(m, U)$ とそれに対する実行可能なスケジュール $(M, X)$ に対し $\forall i\pause A_i = 0$ と置くと, $\tau(M, X) = \tau^\prime(m, U, S)$ となるような実行可能な自由スケジュール $S$ が存在する.この変換を行う写像を次の補題で定義する.
\begin{lemma}\label{lemma:bs-to-fsabs}
  BS の入力 $(m, U)$ に対し $(M, X)$ を任意のスケジュールとし, $A$ を $m$ 個の 0 からなる列とする.以下の条件を満たす自由スケジュール $S$ が存在する.
  \begin{enumerate}
  \item $(M, X)$ が入力 $(m, U)$ に対し実行可能であるならば $S$ も入力 $(m, U, A)$ に対し実行可能である.
  \item $(M, X)$ が実行可能でないならば $S$ も実行可能ではない.
  \item $(M, X)$ が実行可能なとき,$\tau(M, X) = \tau^\prime(S)$.
  \end{enumerate}
\end{lemma}
\begin{proof}
  $f$ を次のように定義する.$X$ の要素が $n$ 個のとき, $S$ をの要素を
  \begin{equation}
    (\alpha_{i,j}, \beta_{i,j}) = (x_j, M_{i,j})
  \end{equation}
  とする. $S$ において明らかに $n_i = n$ となり,
  \begin{align}
    \sum_j^{n_i} \alpha_{i,j} &= \sum_j^n x_j \\
                              &= 1 - A_i \\
                              &= 1
  \end{align}
  となる.したがって \defref{fsabs-feasible-schedule} の条件 1 は常に満たされる.さらに, $S$ に対し明らかに
  \begin{equation}
    \rho_{i, j} = \begin{cases}
      \sigma_{i - 1, j - 1}
    \end{cases}
  \end{equation}
  及び
  \begin{equation}
    \beta_{i,j} = M_{i,j}
  \end{equation}
  が成り立つ.したがって,\defref{bs-feasible-schedule} の条件 1 と \defref{fsabs-feasible-schedule} の条件 2 は同値である.同様に, $\chi_{i,j} \cap \chi_{i^\prime, j^\prime} \neq \emptyset$ は $j = j^\prime$ と同値であるため,\defref{bs-feasible-schedule} の条件 2 と \defref{fsabs-feasible-schedule} の条件 3 も同値である.最後に, $\sigma_{i,j} \leq \rho_{i^\prime, j^\prime}$ ならば $j < j^\prime$ となり, $t^\prime(\rho_{i^\prime, j^\prime}) = t^\prime(\sigma_{i^\prime, j^\prime - 1})$ となるため, $j^\prime = j - 1$ と仮定すると \defref{fsabs-feasible-schedule} の条件 4 は \defref{bs-feasible-schedule} の条件 3 を含意する.他方で,帰納法を用いることでその逆も示すことができる.また, \defref{fsabs-feasible-schedule} の条件 5 と \defref{bs-feasible-schedule} の条件 4 も自明に同値となる.

  以上の議論を以て,主張 1 と 2 が正しいと言える.主張 3 に関しては $t^\prime_i(1) = t_i(X, M)$ が成り立つことで容易に満たされることが分かる.
\end{proof}
\lemref{bs-to-fsabs} は FSABS の解法アルゴリズムで用いることとなる.

続いて,いくつかの調整を加えることで任意の実行可能な自由スケジュールを実行可能なスケジュールへ変換できることを示す.その準備として, FSABS と BS の入力となる $U$ に対し,元 $u_{-1} = 0$ が常に存在すると仮定する.
\begin{lemma}\label{lemma:fsabs-to-bs}
  FSABS の入力 $(m, U, A)$ に対し $S$ を任意の実行可能な自由スケジュールとする.以下の条件を満たすスケジュール $(M, X)$ が存在する.
  \begin{enumerate}
  \item $(M, X)$ は実行可能である.
  \item 各 $A_i$ に対し, $\sum_{j=1}^{k_i}x_j = A_i$ となる $k_i$ が存在する.ただし $A_i = 0$ の場合は $k_i = 0$ とする.
  \item $t_{i,k_i}(M, X) = 0$.
  \item $\tau(M, X) = \tau^\prime(S)$.
  \end{enumerate}
\end{lemma}
\begin{proof}
  $(M, X)$ を次のように構成する.まず, $S$ に対する全ての $\rho_{i,j}$ 及び $\sigma_{i,j}$ を実数直線の区間 $[0, 1]$ 上に並べ,重複を除いた上でそれらの点による分割を $X$ の値とする.この構成により,始点が $A_i$ となる小区間 $x_{k_i}$ は $\rho_{i,0}$ によって与えられ,必ず存在するので主張の条件 2 が容易に満たされることが分かる. $X$ の元の数を $n$ とすると $M$ の次元も $m\times n$ と決まる.次に, $x_k$ に対する小区間を
  \begin{equation}
    \chi^\prime_k = \Big(\sum_{j=1}^{k-1} x_j, \sum_{j=1}^{k} x_j\Big)
  \end{equation}
  とすると, $\chi^\prime_k \subset \chi_{i,k^\prime}$ のときに $M(i,k) = \beta_{i,k^\prime}$ とする.ただし, $X$ の構成方法より $\chi^\prime_k$ は $\chi_{i,k^\prime}$ に完全に含まれるか,共通部分を持たないことが保証されることに注意する.一方で $j \leq k_i$ に対し $M(i,j) = -1$ とすると
  \begin{equation}
    t_{i,k_i}(M, X) = \sum_{j = 1}^{k_i} x_j \times 0 = 0
  \end{equation}
  となるので条件 3 も容易に満たされる.

  \defref{fsabs-feasible-schedule} より全ての $\beta_{i,j} \neq 0$ に対し $\sigma_{i^\prime, j^\prime} = \rho_{i,j}$ かつ $\beta_{i^\prime, j^\prime} = \beta_{i,j}$ を満たす $i^\prime$ と $j^\prime$ が存在するが, $X$ の構成より $\sigma_{i^\prime, j^\prime} = \rho_{i,j}$ を満たすものは隣同士の小区間となるので \defref{bs-feasible-schedule} の条件 1 が満たされる.同様な考え方を用いることで \defref{bs-feasible-schedule} の他の条件も \defref{fsabs-feasible-schedule} の条件より容易に導出される.したがって $(M, X)$ も実行可能である.また, $t_i(M, X) = t^\prime_{i,n_i}(1)$ となることを利用すれば,主張の条件 4 が満たされる.
\end{proof}

以降は FSABS の下界について論じる. \lemref{lower-bound-bs} と同様に以下の補題で FSABS に対する下界を定義できる.
\begin{lemma}\label{lemma:fsabs-lower-bound-absolute}
  \begin{align}
    \bar\tau^{\prime}(m, U, A) &\geq T(m, U, A) \\
                      &= 1 - \frac{1}{m}\sum_{j = 1}^b (1 - u_j) - \frac{1}{m}\sum_{i = 1}^{m} A_i \\
                      &= T(m, U) - \frac{1}{m}\sum_{i = 1}^{m} A_i
  \end{align}
\end{lemma}
\begin{proof}
  \lemref{lower-bound-bs} と同様に $T(m, U, A)$ は全員の移動時間の平均値を表すので,最大値が平均値以上であることから主張が成立する.
\end{proof}
% また,同じく \lemref{lower-bound-bs-bike} と同様に次の主張も容易に成り立つ.
% \begin{lemma}\label{lemma:fsabs-lower-bound-bike}
%   \begin{equation}
%     \bar\tau^\prime(m, U, A) \geq u_b
%   \end{equation}
% \end{lemma}

次に新しい形での下界を導入する.人を増やすことによって到着時間が早くなることは直感的に考えにくく, BS においては $T(m, U)$ の $m$ に対する単調増加性を示すことによってそれを形式的に証明できる.以下の補題では同じ考え方を FSABS について証明する.ただし $A_{:k}$ を $A_k$ までを含んだ列とする\footnote{このような表記は Python 言語に基いており, $U$ の中で $b-r$ 未満の番号を持つ元を含めた部分列という意味である.}.

\begin{lemma}\label{lemma:fsabs-lower-bound-recursive}
  $m > b$ とする.このとき
  \begin{equation}
    \bar\tau^{\prime}(m, U, A) \geq \bar\tau^\prime(m - 1, U, A_{:m-1})
  \end{equation}
\end{lemma}
\begin{proof}
  背理法を用いて $\bar\tau^{\prime}(m, U, A) < \bar\tau^\prime(m - 1, U, A_{:m-1})$ であると仮定し,$S$ を $(m, U, A)$ に対する最適な自由スケジュールとし, $(M, X)$ を\lemref{fsabs-to-bs} により得られた $S$ に対応するスケジュールとする.また,新たな記号を導入せずに本証明に限り $\tau^\prime(S)$ の代わりに $\tau^\prime(M, X, A)$ と書くことにする.

  人 $m$ がはじめて自転車に乗る小区間を $x_i$ とし,その始点において人 $m$ が乗る自転車 $u_j$ に注目する.その自転車には 人 $\alpha$ が小区間 $x_{i - 1}$ で乗っており,事前に $x_i$ の始点まで運んでくれたはずなので人 $\alpha$ に引き続き $x_i$ で乗ってもらう.そうすることで自転車 $u_j$ は少なくともスケジュール通りに $x_i$ の終点に到着する (少なくともというのは,乗り換えの都合で $u_j$ が一定時間使用されない可能性があるのに対し,人 $\alpha$ がずっと乗ることでその時間が省けるということを意味する). しかし,元々人 $\alpha$ は $x_i$ で $u_j$ を使わないことになっている.もし人 $\alpha$ が $x_i$ で $u_k$ の自転車に乗る予定だったならば,今度は同じ議論を $u_k$ を運んでくれた人に対して適用する.それを繰り返すと,いずれ $x_i$ を徒歩で移動する予定だった人 $\beta$ にたどり着く (なぜなら自転車の数が $b \leq m - 1$ であるからである).その人は $x_{i - 1}$ で何かの自転車に乗っている前提だが,その自転車を引き続き使えば良い.

ここまでの処理を施すと,元々 $x_i$ の終点にとある時刻に到着すべきだった人達が人 $m$ 以外全員揃い,誰かが早く到着したとしてもそこで待てば良い\footnote{自由スケジュールの定義では「待つ」という行動は実現できないが, \lemref{fsabs-to-bs} と同様なやり方で容易に追加するができる.}.ただし,元々と違うのは $x_{i + 1}$ 以降の役割が入れ替わっており,上記で言う人 $\alpha$ が人 $m$ の役割を果たすようになっている.「役割を果たす」というのは $x_{i + 1}$ 以降のスケジュールを入れ替え,人 $\alpha$ が人 $m$ のスケジュールを取れば良い. しかし人 $m$ を無視することによって一人役割が余っている人がいる.

上記の議論では 「事前に」 という条件を付けているが,これはつまり自転車の運搬を遡るにつれて,ある自転車を運んだ人がその自転車を使う人よりも早く $x_i$ の始点に到着して, $x_i$ での移動を始めているという前提である.そうでないと自転車の使用者が自転車の到着よりも先に「乗る」ことになるが,それは実行可能解の定義に矛盾する.ここで元々 $x_i$ で歩行する予定だった上記の人 $\beta$ に注目する.人 $\beta$ は $x_i$ で現在自転車を使用し,違う人の役割を担っている.しかし元々のスケジュールでは歩行する予定で,少なくとも人 $m$ と同じタイミングか,それより早く $x_i$ の始点に到着し次の移動に移る.もし人 $\beta$ が人 $m$ と同時出発だったのであれば,人 $m$ は初期位置を変えずそのまま歩けば良い.他方でもし人 $m$ より早い場合は,人 $m$ の初期位置を適切にずらすことで,人 $m$ が元々の人 $\beta$ の到着時刻に $x_i$ の終点に到着するように変更できる.

この操作を施すことによって元々のスケジュールにかかる時間を保ちながら,人 $m$ が最初に自転車に乗る区間を前にずらすことができる.これを最悪 $n$ 回 (小区間の数) 繰り返せば,人 $m$ が自転車を使わないスケジュール $(M\prime, X\prime)$ が得られ, $\tau^\prime(M^\prime, X^\prime, A) \leq \tau^\prime(M, X, A) = \bar\tau^{\prime}(m, U, A)$ が満たされる.しかし人 $m$ が自転車を使わなければ, \propref{vertical-composition} と同様に $\tau^\prime(M^\prime, X^\prime, A) \geq \bar\tau^\prime(m - 1, U, A_{:m-1})$ が成り立つので
\begin{align}
  \bar\tau^\prime(m - 1, U, A_{:m-1}) &\leq \tau^\prime(M^\prime, X^\prime, A) \\
                                      &\leq \tau^\prime(M, X, A) \\
                                      &= \bar\tau^\prime(m, U, A)
\end{align}
より矛盾が生じる.
\end{proof}
この結果から以下の系が容易に得られる.
\begin{corollary}\label{corollary:fsabs-lower-bound-bs}
  $i \leq m - f$ に対し $A_i = 0$ とする.
  \begin{equation}
    \bar\tau^{\prime}(m, U, A) \geq \bar\tau(m - f, U).
  \end{equation}
\end{corollary}

以上の議論を次の定理にまとめる.
\begin{theorem}\label{theorem:fsabs-lower-bound}
  $(m, U, A)$ を FSABS の入力とする.また $f$ を前方にいる人の数とし, FSABS における最適な到着時間 $\bar\tau^\prime(m, U, A)$ は次の下界により制限される.
  \begin{equation}
    \bar\tau^\prime(m, U, A) \geq \max \begin{cases}
      T^\prime(m, U, A) \\
      T^\prime(m - 1, U, A_{:m-1}) \\
      T^\prime(m - 2, U, A_{:m-2}) \\
      \vdots \\
      T^\prime(m - f + 1, U, A_{:m-f+1}) \\
      \bar\tau(m - f, U)
    \end{cases}
  \end{equation}
\end{theorem}
\begin{proof}
  \lemref{fsabs-lower-bound-recursive} に \lemref{fsabs-lower-bound-absolute} と \corref{fsabs-lower-bound-bs} を適用れば良い.
\end{proof}
\subsection{FSABS を解くアルゴリズム \textsc{Solve-FSABS}}
本節では FSABS を解くアルゴリズム \textsc{Solve-FSABS} を定義し,その計算量と正当性について論じる.まず,アルゴリズムの挙動を \algref{solve-fsabs} に擬似コードで示した.
\IncMargin{0.8em}
\begin{algorithm}
  \caption{\textsc{Solve-FSABS}}\label{alg:solve-fsabs}
  \SetKwInOut{Input}{input}\SetKwInOut{Output}{output}
  \SetKw{Not}{not}
  \SetKw{In}{in}
  \SetKw{And}{and}
  \SetKw{Continue}{continue}
  \SetKwProg{Fn}{Function}{:}{}
  \SetKwFunction{FToNextW}{\textsc{ToNextW}}
  \SetKwFunction{FToNextR}{\textsc{ToNextR}}
  \SetKwFunction{FToEnd}{\textsc{ToEnd}}
  \SetKwFunction{FMove}{\textsc{Move}}
  \SetKwFunction{FSolveFSABS}{\textsc{Solve-FSABS}}
  \SetKwFunction{FSubgroup}{\textsc{Subgroup}}
  \SetKwFunction{FMerge}{\textsc{Merge}}
  \SetKwFunction{FApplyMovement}{\textsc{ApplyMovement}}
  \SetKwFunction{FOnlyRidersLeft}{\textsc{OnlyRidersLeft}}
  \SetKwFunction{FMoveRiders}{\textsc{MoveRiders}}
  \SetKwFunction{FScalePositions}{\textsc{ScalePositions}}
  \SetKwFunction{FScaleSchedule}{\textsc{ScaleSchedule}}
  \SetInd{0.25em}{0.25em}

  \Input{自然数 $m$\\
    昇順に並べた列 $U \in (0, 1)^{b}$\\
    昇順に並べた列 $A \in (-\infty, 1)^{m}$}
  \Output{自由スケジュール $S$}
  \If{$\forall i,\; A_i = 1$}{
    \Return []\;
  }
  $r \gets$ 後方にいるライダーの数\;
  $f \gets$ 前方にいる歩行者の数\;
  \If{$u_b > T(m - f - r, U_{:b-r})$}{\label{alg:line:if-subgroup}
    $m_k \gets$ \FSubgroup{$m$, $U_{:b-r}$}\; \label{alg:line:subgroup}
    $r \gets r + m - m_k$\;
  }
  $(t_1, d_1) \gets (\infty, \infty)$\;
  $(t_2, d_2) \gets (\infty, \infty)$\;
  \If{$f > 0$} {
    $(t_1, d_1) \gets$ \FToNextW{$m$, $U$, $A$, $r$}\; \label{alg:line:toNextW}
  }
  \Else {
    $(t_1, d_1) \gets$ \FToEnd{$m$, $U$, $A$, $r$}\; \label{alg:line:toEnd}
  }
  groupT $\gets T(m - f - r, U_{:b-r})$\;
  \If{$r > 0$ \And $u_{b - r} <$ \upshape{groupT}}{
    $(t_2, d_2) \gets$ \FToNextR{$m$, $U$, $A$, $r$}\; \label{alg:line:toNextR}
  }
  $(t, d) \gets$ (NIL, NIL)\;
  \If{$t_2 < t_1$}{
    $(t, d) \gets (t_2, d_2)$\;
  }
  \Else{
    $(t, d) \gets (t_1, d_1)$\;
  }
  $S \gets$ \FMove{$t$, $m$, $U$, $A$}\; \label{alg:line:find-schedule}
  $A^{(1)} \gets$ \FApplyMovement{$S$, $A$}\; \label{alg:line:apply-movement}
  $S^{(1)} \gets$ []\;
  \If{$\forall i > r \pause A^{(1)}_i = 1$}{ \label{alg:line:check-only-riders}
    $S^{(1)} \gets$ \FMoveRiders{$m$, $U$, $A^{(1)}$}\; \label{alg:line:move-riders}
  }
  \ElseIf{$d < 1$}{ \label{alg:line:check-not-arrived-yet}
    $A^{(2)} \gets \frac{A^{(1)} - d}{1 - d}$\;
    $S^{(2)} \gets$ \FSolveFSABS{$m$, $U$, $A^{(2)}$}\; \label{alg:line:find-recursive-schedule}
    $S^{(1)} \gets$ \FScaleSchedule{$S^{(2)}$, $1 - d$}\; \label{alg:line:rescale-recursive}
  }
  \Return{$S + S^{(1)}$}\; \label{alg:line:merge-schedules}
\end{algorithm}
\DecMargin{0.8em}

\textsc{Solve-FSABS} の挙動を説明する前にいくつかの補助的な概念を導入する.
\begin{enumerate}
\item グループ:出発点にいる人と自転車のまとまり.グループの移動は BS のインスタンスとして計算することができる.グループの人に対し $A_i = 0$ が成り立つ.
\item 後方ライダー:出発点より後ろにおり,ずっと同じ自転車に乗っている人.ライダーに対し $A_i < 0$ が成り立つ.
\item 前方歩行者:出発点より前におり,ずっと歩いている人.歩行者に対し $A_i > 0$ が成り立つ.
\end{enumerate}
これらを踏まえた上で \textsc{Solve-FSABS} の入力に対する次の制約を述べる.
\begin{condition}\label{condition:riders-order}
  ライダーとなる ($A_i < 0$ となる) 人 $i$ は $u_{b - i}$ の自転車を使っていなければならない.つまり,ライダーとなる人達は常に一番遅い自転車に乗っており,さらに遅ければ遅いほど後ろにいなければならない.
\end{condition}
\textsc{Solve-FSABS} は再帰的なアルゴリズムであるが,呼び出すときには各人が必ず上記のいずれかの分類に含まれる.なお,最初に呼び出すときにはライダーがいないことに注意する.

続いて \textsc{Solve-FSABS} の概ねの挙動を説明する.
\begin{enumerate}
\item 全員を前進させ,出発点にいるグループが後ろから追いつてくるライダーもしくは前方にいる歩行者のいずれか早い方と合流する点までの距離と移動スケジュールを計算する.なお,もし到着点までに合流できない場合は全区間を移動させる.
\item グループが動いた時間だけ他のライダーや歩行者を動かす.
\item 合流地点を新たな出発点として考え,後方ライダーや前方歩行者の相対的な位置を再計算した上で再帰的に \textsc{Solve-FSABS} を呼び出し FSABS を解く.この祭,合流によって後方ライダもしくは前方歩行者が少なくとも一人グループに吸収されるので入力がより簡単になることに注意する.
\item 新たに解いた問題のスケジュールを最初に計算したスケジュールと合併させる.
\end{enumerate}

最初にアルゴリズムを呼び出すときにライダーがいないが,ステップ 1 ではライダーを考慮する必要がある. \textsc{Solve-FSABS} は再帰的なアルゴリズムなので,ライダーがどのように登場するかを後ほどの議論に任せ,以降ではライダーがいる前提での挙動について論じる.

ステップ 1 の通り,グループを移動させ,ライダーもしくは歩行者と合流させたい.なお,グループに入る自転車はライダーの自転車を含まないので, \condref{riders-order} を活用しグループの自転車を列 $U_{:b-r}$ として表すことができることに注意する. \lemref{fsabs-lower-bound-absolute} を満たすためには到着点に到達していない人が全員常に動いていなければならないので,合流する時にグループも歩行者 (ライダー) も同時に合流点に到着する必要がある.しかし,グループが持っている自転車の速度によっては,一定区間動かしたときに全員が同時に到着しない場合があり得る.幸いなことに,そのような場合には \lemref{bs-subgroup} より同時到着を実現できる部分グループの存在が保証されるので,その部分グループを新たに「グループ」として考え,余った人と自転車をライダーとして考える.ただしその祭に入力変数 $r$ を変更する必要があることに注意する.アルゴリズムの \lineref{if-subgroup} の条件分岐がこの部分に対応する.

部分グループの採用による調整を行った後,次に合流できる人を定める.まず,もし先に歩行者がまだいるならば,その歩行者と合流する点を求める.上述の通り合流点にグループも歩行者も同時に到着しなければならないので,合流点を $x_1$ とするとグループも歩行者もそれぞれ同じ時間でそこ到達しなければならない. \claimref{bs-scalable} より $\bar\tau(m - f - r, U) = T(m - f - r, U)$ をグループの速度として考えることができるので,出発点に一番近い歩行者を人 $p$ とするとそれぞれが合流点までの移動にかかる時間を以下の式で表すことができる.
\begin{equation}
  t_1 = (x_1 - A_p) \times 1 = x_1T(m - f - r, U_{:b-r})
\end{equation}
$x_1$ について解くと合流点が求まる.
\begin{equation}
  x_1 = \frac{A_p}{1 - T(m - f - r, U_{:b-r})}
\end{equation}
もし $x_1 \geq 1$,つまり合流点が到着点以降になるのであれば, $x_1 = 1$ と置いて到着点までの移動だけを考える.この実際の移動距離を $d_1 = \min \{x_1, 1\}$ と置く.以上の計算を行うのが \lineref{toNextW} の関数 \textsc{toNextW} の役割である.他方でもし先に歩行者がいなければ, $x_1 = 1$ の場合と同様に計算すれば良い.その場合を賄うのが \lineref{toEnd} の関数 \textsc{toEnd} の役割である.

次に,もし後ろにライダーがいて,且つライダーがグループより速く動くのであれば,ライダーと先に合流できないかを調べる必要がある.その際には上記と同様な方法で合流時間
\begin{equation}
  t_2 = (x_2 - A_r)u_{b - r} = x_2T(m - f - r, U_{:b-r})
\end{equation}
から合流点 $x_2$ について解くことで値が求まる.
\begin{equation}
  x_2 = \frac{A_ru_{b - r}}{u_{b-r} - T(m - f - r, U_{:b-r})}
\end{equation}
この部分が \lineref{toNextR} の関数 \textsc{toNextR} の役割である.最後に,ライダーと先に合流するか,歩行者と先に合流するか,両方よりも先に到着点に到達するかを時間的に比較した上で,一番早くできる行動を採用して,それにかかる移動を時間 $t$ 及び距離 $d$ で表す.

移動距離を決めてから次にその移動を実行するためのスケジュール $S$ を \lineref{find-schedule} で求める.手続き \textsc{Move} は各人 $i$ に対し次のようなスケジュールを構成し,それらをまとめたものを返す.
\begin{itemize}
\item 人 $i$ が既に到着点にいる ($A_i = 1$) ならば,その人の自由スケジュールを空列とする.
\item 人 $i$ がライダー ($0 < i \leq r$) ならば,その人の自由スケジュールを一つの順序対 $(tu_{b-i}, b-i)$ からなる配列とする.
\item 人 $i$ がグループの人 ($r < i < m - f$) ならば,グループ全体のスケジュール $(M, X)$ を BS の解として求め, \claimref{bs-scalable} を用いて距離 $\frac{t}{T(m - f - r, U_{:b-r})}$ に合わせた上で, \lemref{bs-to-fsabs} により与えられる  $(M, X)$ に相当する自由スケジュール $S^\prime$ を組み立て,人 $i$ の自由スケジュールを $S^\prime_i$ とする.
\item 人 $i$ が歩行者 ($m - f < i < m$) ならば,その人の自由スケジュールを一つの順序対 $(\min \{ 1 - A_i, t \}, 0)$ からなる配列とする.
\end{itemize}
時間 $t$ 分の移動スケジュールを求めた上で,その移動を $A$ に対する操作として施し, $A$ の値にそれぞれの人の移動距離を加算したものを $A^{(1)}$ に格納する (\lineref{apply-movement}).

次に移動後の状態を改めて分析した上で次の挙動を決める.もしグループの移動距離が全区間ならば,前方の歩行者とグループが必ず到着点に到達したと解釈できる.その祭,$i > r$ に対し $A^{(1)}_i$ の値は 1 となる.しかし,もしライダーがいる場合にはライダーがまだ到着点に到達していない可能性があり,ライダーのスケジュールにそれぞれが到着するまでの分を追加しなければならない.この場合に対応するのが \lineref{check-only-riders} の条件分岐である.

他方で,もしグループが全区間を移動していないのであれば,少なくともグループのメンバーはまだ到着点に到達していない.その祭,全員の相対位置をスケールした上で,残りの区間に対し新たに \textsc{Solve-FSABS} を用いて問題を解く.ただし,相対位置を計算するときには以下の式を用いる.
\begin{equation}
  A^{(2)}_i = \begin{cases}
    \frac{A^{(1)}_i - d}{1 - d} & A^{(1)}_i < 1 \\
    A^{(1)}_i & A^{(1)}_i = 1
  \end{cases}
\end{equation}
この処理を施し, \textsc{Solve-FSABS} を再び呼び出した結果が自由スケジュール $S^{(2)}$ となり,アルゴリズムが正当であると仮定すると $S^{(2)}$ の移動を施せば全員が到着点に到達する.しかし, $S^{(2)}$ では移動区間を $[0, 1]$ としているので,元の問題に適用するには全員の移動を区間 $[d, 1 -d]$ に合うようにスケールしなければならない.すなわち, $S^{(2)}$ の全ての順序対に対し以下の順序対
\begin{equation}
  (\alpha^{(1)}_{i, j}, \beta^{(1)}_{i,j}) = ((1 - d)\alpha^{(2)}_{i,j}, \beta^{(1)}_{i,j})
\end{equation}
を求め,それらを $S^{(1)}$ にまとめる.この処理が \lineref{check-not-arrived-yet} の条件分岐に相当する.

最後に,合流までのスケジュール $S$ と合流後のスケジュール $S^{(1)}$ を組み合わせたものを返せば良い.ただし,組み合わせるとはそれぞれの $i$ に対し配列 $S_i$ 及び $S^{(1)}_i$ を連結させる処理とする.
\subsection{\textsc{Solve-FSABS} の正当性及び計算時間}
TODO
\section{結論と今後の課題}
TODO
書かなくても良いかも?
% \begin{appendices}
% \section{$S$ から $(M, X)$ を生成する手続き}
% \label{appendix:s-to-mx}
% TODO
% \end{appendices}

%%%%%%%%%
% ダミーの文章、および定規を生成する記述です。原稿作成の際は消去下さい
% \fgruler{upperleft}{0cm}{0cm}
% \Blindtext
% \fgruler{upperleft}{0cm}{0cm}
% \Blindtext
%%%%%%%%%


\end{document}


自律移動ロボットは,
多くの製造,倉庫保管,およびロジスティクスのアプリケーションで
ますます使用されている.
最近では特に,人間と協調して近接して動作することを目的とした,
いわゆるコボット(協働ロボット)の展開への関心が高まっている[1,3,23].
このようなコボットは人間によって制御されるが,
一緒に作業する人間の能力を増強・強化することを目的としている.

小文では,輸送問題へのコボットの応用について考える.
この問題では,
協調する自律移動エージェント(人間またはロボット)は
輸送コボット(小文では自転車と呼ばれる)の使用時に
エージェントの速度を上げることで輸送コボットから支援を受ける.
エージェントは自律的で,能力が等しく,最高速度1で歩くことができる.
自転車は自律的ではなく,自分で移動することはできないが,
エージェントは自転車に乗って移動することができる.
エージェントはいつでも高々$1$台の自転車に乗ることができ,
自転車は高々1人のエージェントを乗せることができる.
自転車$i$に乗っているエージェントは速度$v_i>1$で移動できる;
バイクが異なれば速度も異なるかもしれないことに注意されたい.
自転車は2つの役割を果たす:
エージェントが自分の速度を上げるために利用できるリソースであるが,
輸送する必要のある商品でもある.
$m$人のエージェントと$b$台の自転車があるとし,
自転車$1,2,\ldots,b$の速度はそれぞれ$v_1\ge v_2\ge\cdots\ge v_b>1$
であるとする.
また,エージェントと自転車の初期位置は全て単位区間$[0,1]$にあるとし,
エージェント$i$ ($1\le i\le m$)の初期位置は$A_i$であり,
自転車の初期位置は$0$であるとする.
ただし,$A_i=0$であるエージェント$i$が少なくとも$b+1$人いると仮定する.
エージェントの目標は,
エージェントと自転車の最後の到着時間が最小となるように,
自転車を輸送しながら目的地点$1$へ到達することである.

Czyzowiczら[The bike sharing problem]は,
全てのエージェント$i$に対して$A_i=0$である問題
(この問題を自転車共有問題, 略してBS問題と呼ぶ)の最適解を求める
多項式時間アルゴリズムを開発した.
小文では,$A_i>0$であるエージェント$i$が存在することを許した問題
(これを前方に人がいることを許した自転車共有問題,略してFSABS問題と呼ぶ)の
最適解を求める多項式時間アルゴリズムを提案する.

% (関連研究は省略しますが,参考文献に文献The bike sharing problemの
% [2,4--12,14,\allowbreak 15,\allowbreak 17--19,21,22]を
% リストしておいて下さい.
% 引用せずに参考文献に載せるには\verb+\nocite+を用いること.)

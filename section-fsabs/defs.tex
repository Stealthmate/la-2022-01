FSABS を定義する前に 2 つの補助的概念を導入する.

まず,各人の初期位置を表すための配列 $A \in {(0, 1)}^{m}$ を考え, $A$ の要素が全員分あることに注意する.理論的には前方にいる人の初期位置だけで十分であるが,以降で紹介するアルゴリズムの動作の都合上,全員分の初期位置を用意する方が扱いやすいため,ここでは出発点にいる人の初期位置を 0 とし,その人達の初期位置を含めた $A$ を考える.

BS の出力は各人が各小区間で使用した自転車の番号を表す行列 $M$ と各小区間の長さを表すベクトル $X$ と定義した. FSABS の出力にもこのような形を採用することはできるが,上記同様に解法アルゴリズムの動作の都合上,次のような形の出力を考える. $m$ 個の要素からなる配列 $S$ を考え,それぞれの要素がそれぞれの人に対応するとする.人 $i$ に対して $S_i$ の値は $n_i$ 次元配列であり,それの各要素が $(\alpha_{i,j}, \beta_{i,j})$ のような順序対であり,人 $i$ が自転車 $\beta_{i,j}$ で距離 $\alpha_{i,j}$ を連続移動したことを意味する.ただし,歩行したならば $\beta_{i,j} = 0$ と置き, $0 < j < n_i$ とする.このような $S$ を自由スケジュールと呼ぶ.また, $S_i$ のことを $S$ の行と呼ぶ.

$S_i$ の各要素に対応する $[0, 1]$ 上の始点と終点をそれぞれ $\rho_{i,j}$ 及び $\sigma_{i,j}$ で表すことができる.
\begin{align}
  \rho_{i,j} &= A_i + \sum_{k=1}^{k < j}\alpha_{i, k} \\
  \sigma_{i,j} &= \rho_{i,j} + \alpha_{i, k}
\end{align}
これらをまとめて,$\alpha_{i,j}$ に対応する小区間を $\chi_{i,j} = (\rho_{i,j}, \sigma_{i,j})$ と表す.

$S$ において人 $i$ が点 $x \geq A_i$ に到達する時間を $t^{\prime}_i(x)$ とする. $t^{\prime}_i(x)$ の値は $S_i$ の要素の線形和として求めることができるが,複雑な表記を必要とするためここでは省略する.全体の到着時間は BS と同様にそれぞれの人の到着時間の最大値となるので, $S$ における到着時間を
\begin{equation}
  \tau^\prime(S) = \max_i t^{\prime}_i(1)
\end{equation}
と定義できる.また, $S$ において人 $i$ が移動した距離を
\begin{equation}
  L(S_i) = \sum_{j=1}^{n_i} \alpha_{i,j}
\end{equation}
と表す.

次に $S$ が実行可能解であるために \textcite{czyzowicz} と同様な条件を述べる.
\begin{definition}\label{definition:fsabs-feasible-schedule}
  入力 $(m, U, A)$ と上述の構造を持つ自由スケジュール $S$ に対し,以下が成り立つとき且つそのときに限り $S$ を実行可能な自由スケジュールと呼ぶ.
  \begin{enumerate}
  \item $(\forall i\st 1 \leq i \leq m)\quad L(S_i) = 1 - A_i$.
  \item $\beta_{i,j} \neq 0$ であるならば $\sigma_{i^\prime, j^\prime} = \rho_{i,j} \text{ かつ } \beta_{i,j} = \beta_{i^\prime, j^\prime}$ を満たす $i^\prime$, $j^\prime$ が存在する.
  \item $\beta_{i,j} \neq 0$ であるならば $\chi_{i,j} \cap \chi_{i^\prime,j^\prime} \neq \emptyset$ となるような $i^\prime$ 及び $j^\prime$ に対し $\beta_{i,j} \neq \beta_{i^\prime,j ^\prime}$.
  \item $\beta_{i,j} = \beta_{i^\prime, j^\prime}$ 及び $\sigma_{i,j} \leq \rho_{i^\prime, j^{\prime}}$ ならば $t^{\prime}_i(\sigma_{i, j}) \leq t^{\prime}_{i^\prime}(\rho_{i^\prime, j^\prime})$.
  \item $\forall j \st 1 \leq j \leq b\pause \exists i \st \beta_{i,n_i} = j$.
  \end{enumerate}
\end{definition}

以上の内容をまとめて, FSABS を次のように定義する.
\begin{problem}
  \problemtitle{FSABS}
  \probleminput{
    $m \in \N$: 人の数. \newline
    $U \in (0, 1)^{b}$: 自転車の速度の逆数を格納した昇順配列. \newline
    $A \in (0, 1)^{m}$: 各人の初期値を格納した昇順配列.
  }
  \problemoutput{
    $S$: 実行可能な自由スケジュール.
  }
  \problemobjective{
    $\tau^\prime(S)$ を最小化すること.
  }
\end{problem}
また,以降の議論では具体的な出力に関係なく,とある入力に対する最適な到着時間を $\bar\tau^\prime(m, U, A)$ と表す.

解法アルゴリズムでは人をグループに分け,それぞれの移動を独立に考えてからそれぞれの自由スケジュールを組み合わせるという操作を頻用する.この操作を「縦に組み合わせる」と名付け,以下の命題でその正当性を保証する.
\begin{proposition}\label{proposition:vertical-composition}
  $(m_1, U_1 = [u_1, u_2, \ldots, u_{b_1}], A_1)$ 及び $(m_2, U_2 = [u_{b_1+1}, u_{b_1+2}, \ldots, u_{b_1+b_2}], A_2)$ をそれぞれ FSABS の入力とし, $S^{(1)}$ 及び $S^{(2)}$ をそれぞれに対する実行可能解とする. $A_1$ 及び $A_2$ の要素を $A_3$ にまとめ,昇順に並べることで新しくできた人の番号を用いて $S^{(1)}$ 及び $S^{(2)}$ の行を適切な順番に並べた $S^{(3)}$ を構成する.このとき, $S^{(3)}$ は入力 $(m_1 + m_2, U_1 + U_2, A^{(3)}$ に対する実行可能解となり,その到着時間が
  \begin{equation}
    \tau^\prime(S^{(3)}) = \max \{ \tau^\prime(S^{(1)}), \tau^\prime(S^{(2)}) \}
  \end{equation}
  となる.
\end{proposition}
\begin{proof}
  $S^{(1)}$ 及び $S^{(2)}$ は独立に実行可能解であり,且つお互いの自転車を共有することがないので \defref{fsabs-feasible-schedule} の条件は自明に満たされる.
\end{proof}

最後に, FSABS が線形スケール化可能であることを主張しておく.
\begin{claim}\label{claim:fsabs-scalable}
  FSABS の入力 $(m, U, A)$ に対し $S$ が実行可能解であるとする. \defref{fsabs-feasible-schedule} や FSABS の入力に対する条件を適切に修正し,区間 $[0, 1]$ ではなく $[0, a]$ に対し問題を定義したとする.以下で与えられる $A^\prime$ 及び $S^\prime$ に対し, $S^\prime$ は入力 $(m, U, A^\prime$ に対する実行可能解となり, $\tau^\prime(S) = a\tau^\prime(S^\prime)$ となる.
  \begin{align}
    A^\prime_i &= aA_i \\
    (\alpha^\prime_{i,j},\beta^\prime_{i,j}) &= (a\alpha_{i,j},\beta_{i,j})
  \end{align}
  ただし, $(\alpha_{i,j},\beta_{i,j})$ を $S$ の元とし, $(\alpha^\prime_{i,j},\beta^\prime_{i,j})$ を $S^\prime$ の元とする.
\end{claim}
本節では FSABS と BS の解の形の互換性について説明した上で, FSABS の下界を定める.

問題の定義上, BS のインスタンス集合は意味的に FSABS のインスタンス集合の部分集合である.すなわち,任意の BS のインスタンス $(m, U)$ とそれに対する実行可能なスケジュール $(M, X)$ に対し $\forall i\pause A_i = 0$ と置くと, $\tau(M, X) = \tau^\prime(m, U, S)$ となるような実行可能な自由スケジュール $S$ が存在する.この変換を行う写像を次の補題で定義する.
\begin{lemma}
  BS の入力 $(m, U)$ に対し $(M, X)$ を任意のスケジュールとすし, $A$ を $m$ 個の 0 からなる列とする.以下の条件を満たす写像 $f: (M, X) \mapsto S$ は存在する.
  \begin{enumerate}
  \item $(M, X)$ が入力 $(m, U)$ に対し実行可能であるならば自由スケジュール $S = f(M, X)$ も入力 $(m, U, A)$ に対し実行可能である.
  \item $(M, X)$ が実行可能でないならば自由スケジュール $S = f(M, X)$ も実行可能ではない.
  \item $(M, X)$ が実行可能なとき,$\tau(M, X) = \tau^\prime(f(M, X))$.
  \end{enumerate}
\end{lemma}
\begin{proof}
  $f$ を次のように定義する.$X$ の要素が $n$ 個のとき, $S = f{(M, X)}$ をの要素を
  \begin{equation}
    (\alpha_{i,j}, \beta_{i,j}) = (x_j, M_{i,j})
  \end{equation}
  とする. $S$ において明らかに $n_i = n$ となり,
  \begin{align}
    \sum_j^{n_i} \alpha_{i,j} &= \sum_j^n x_j \\
                              &= 1 - A_i \\
                              &= 1
  \end{align}
  となる.したがって \defref{fsabs-feasible-schedule} の条件 1 は常に満たされる.さらに, $S$ に対し明らかに
  \begin{equation}
    \rho_{i, j} = \begin{cases}
      \sigma_{i - 1, j - 1}
    \end{cases}
  \end{equation}
  及び
  \begin{equation}
    \beta_{i,j} = M_{i,j}
  \end{equation}
  が成り立つ.したがって,\defref{bs-feasible-schedule} の条件 1 と \defref{fsabs-feasible-schedule} の条件 2 は同値である.同様に, $\chi_{i,j} \cap \chi_{i^\prime, j^\prime} \neq \emptyset$ は $j = j^\prime$ と同値であるため,\defref{bs-feasible-schedule} の条件 2 と \defref{fsabs-feasible-schedule} の条件 3 も同値である.最後に, $\sigma_{i,j} \leq \rho_{i^\prime, j^\prime}$ ならば $j < j^\prime$ となり, $t^\prime(\rho_{i^\prime, j^\prime}) = t^\prime(\sigma_{i^\prime, j^\prime - 1})$ となるため, $j^\prime = j - 1$ と仮定すると \defref{fsabs-feasible-schedule} の条件 4 は \defref{bs-feasible-schedule} の条件 3 を含意する.他方で,帰納法を用いることでその逆も示すことができる.

  以上の議論を以て,主張 1 と 2 が正しいと言える.主張 3 に関しては $t^\prime_i(1) = t_i(X, M)$ が成り立つことで容易に満たされることが分かる.
\end{proof}

{\color{red}逆の変換も示すこと}

\lemref{lower-bound-bs} と同様に以下の補題で FSABS に対する下界を定義できる.
\begin{lemma}
  \begin{align}
    \bar\tau^{\prime}(m, U, A) &\geq T(m, U, A) \\
                      &= 1 - \frac{1}{m}\sum_{j = 1}^b (1 - u_j) - \frac{1}{m}\sum_{i = 1}^{m} A_i \\
                      &= T(m, U) - \frac{1}{m}\sum_{i = 1}^{m} A_i
  \end{align}
\end{lemma}
\begin{proof}
  \lemref{lower-bound-bs} と同様に $T(m, U, A)$ は全員の移動時間の平均値を表すので,最大値が平均値以上であることから主張が成立する.
\end{proof}

次に新しい形での下界を導入する.人を増やすことによって到着時間が早くなることは直感的に考えにくく, BS においては $T(m, U)$ の $m$ に対する単調増加性を示すことによってそれを形式的に証明できる.以下の補題では同じ考え方を FSABS について証明する.ただし $A_{:k}$ を $A_k$ までを含んだ列とする.

\begin{lemma}\label{lemma:fsabs-lower-bound-recursive}
  ({\color{red}{補題 or 定理?}}) $m > b$ とする.このとき
  \begin{equation}
    \bar\tau^{\prime}(m, U, A) \geq \bar\tau^\prime(m - 1, U, A_{:m-1})
  \end{equation}
\end{lemma}
\begin{proof}
  背理法を用いて $\bar\tau^{\prime}(m, U, A) < \bar\tau(m - 1, U, A_{:m-1})$ であると仮定し,$(M, X)$ を $(m, U, A)$ に対する最適なスケジュールとする.
  人 $m$ がはじめて自転車に乗る小区間を $x_i$ とし,その始点に乗る自転車 $u_j$ に注目する.その自転車には,
  人 $\alpha$ が小区間 $x_{i - 1}$ で乗っており,事前に $x_i$ の始点まで運んでくれたはずなので人 $\alpha$ に引き続き $x_i$ で乗ってもらう.そうすることで自転車 $u_j$ は少なくともスケジュール通りに $x_i$ の終点に到着する (少なくともというのは,乗り換えの都合で $u_j$ が一定時間使用されない可能性があるのに対し,人 $\alpha$ がずっと乗ることでその時間が省けるということを意味する). しかし,元々人 $\alpha$ は $x_i$ で $u_j$ を使わないことになっている.もし人 $\alpha$ が $x_i$ で $u_k$ の自転車に乗る予定だったならば,今度は同じ議論を $u_k$ を運んでくれた人に対して適用する.それを繰り返すと,いずれ $x_i$ を徒歩で移動する予定だった人 $\beta$ にたどり着く (なぜなら自転車の数が $b \leq m - 1$ だからである).その人は $x_{i - 1}$ で何かの自転車に乗っている前提だが,その自転車を引き続き使えば良い.

ここまでの処理を施すと,元々 $x_i$ の終点にとある時刻に到着すべきだった人達が人 $m$ 以外全員揃い,誰かが早く到着したとしてもそこで待てば良い.ただし,元々と違うのは $x_{i + 1}$ 以降の役割が入れ替わっており,上記で言う人 $\alpha$ が人 $m$ の役割を果たすようになっている.「役割を果たす」というのは $x_{i + 1}$ 以降のスケジュールを入れ替え,人 $\alpha$ が人 $m$ のスケジュールを取れば良い. しかし人 $m$ を無視することによって一人役割が余っている人がいる.

上記の議論では 「事前に」 という条件を付けているが,これはつまり自転車の運搬を遡るにつれて,ある自転車を運んだ人がその自転車を使う人よりも早く $x_i$ の始点に到着して, $x_i$ での移動を始めているという前提である.そうでないと自転車の使用者が自転車の到着よりも先に「乗る」ことになり,おかしい.ここで元々 $x_i$ で歩行する予定だった上記の人 $\beta$ に注目する.人 $\beta$ は $x_i$ で現在自転車を使用し,違う人の役割を担っている.しかし元々のスケジュールでは歩行する予定で,少なくとも人 $m$ と同じタイミングか,それより早く $x_i$ の始点に到着し次の移動に移る.もし人 $\beta$ が人 $m$ と同時出発だったのであれば,人 $m$ は初期位置を変えずそのまま歩けば良い.他方でもし人 $m$ より早い場合は,人 $m$ の初期位置を適切にずらすことで,人 $m$ が元々の人 $\beta$ の到着時刻に $x_i$ の終点に到着するように変更できる.

この操作を施すことによって元々のスケジュールにかかる時間を保ちながら,人 $m$ が最初に自転車に乗る区間を前にずらすことができる.最悪 $n$ 回 (小区間の数) 繰り返せば,人 $m$ が自転車を使わないスケジュール $(M\prime, X\prime)$ が得られ, $\tau(M, X, A) \leq \bar\tau^{\prime}(m, U, A)$ を満たす.しかし人 $m$ が自転車を使わなければ $\tau(M, X, A) \geq \bar\tau(m - 1, U, A_{:m-1})$ が成り立つので, $\bar\tau(m - 1, U, A_{:m-1}) \leq \bar\tau(m - 1, U,A_{:m-1})$ となり矛盾が生じる.
\end{proof}
上記の議論から以下の系が容易に得られる.
\begin{corollary}
  $i \leq m - 1$ に対し $A_i = 0$ とする.
  \begin{equation}
    \bar\tau^{\prime}(m, U, A) \geq \bar\tau(m - 1, U).
  \end{equation}
\end{corollary}
